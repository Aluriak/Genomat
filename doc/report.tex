%%%%%%%%%%%%%%%%%%%%%%%%%
% PACKAGES              %
%%%%%%%%%%%%%%%%%%%%%%%%%
\documentclass[]{report} % book|article|…

\usepackage[utf8x]{inputenc}    % accents
\usepackage{geometry}           % marges
\usepackage[francais]{babel}    % langue
\usepackage{graphicx}           % images
\usepackage{verbatim}           % texte préformaté
\usepackage{fancyhdr}           % fancy
\usepackage{filecontents}       % write file directly
\usepackage{csvsimple}          % csv reader
\usepackage{lastpage}	        % get number of last page
\usepackage{listings}           % source code 
\usepackage{url}                % clickable urls 
\usepackage{float}              % exact placing of figures

% packages for graphics
\usepackage{fancybox}         
\usepackage{pgfplots}        
\usepackage{pgfplotstable}
\usepgfplotslibrary{dateplot}
\usepackage{pgfplots}

\definecolor{Gene0}{RGB}{250, 164, 1}
\definecolor{Gene1}{RGB}{128, 0, 128}
\definecolor{Gene2}{RGB}{255, 0, 0}
\definecolor{Gene3}{RGB}{58, 242, 75}
\definecolor{Gene4}{RGB}{8, 81, 156}
\definecolor{Diversity}{RGB}{0, 0, 0}

\pgfplotscreateplotcyclelist{list}{
{Gene0},
{Gene1},
{Gene2},
{Gene3},
{Gene4},
{Diversity}
}




%%%%%%%%%%%%%%%%%%%%%%%%%
% PRÉAMBULE             %
%%%%%%%%%%%%%%%%%%%%%%%%%
\title{} 
\author{}
% laisser vide pour date de compilation
\date{} 

% FORMAT PAGES         
\pagestyle{fancy} % nom du rendu (définit les lignes suivantes)
        \lhead{M1 BIG} % left head
        \chead{} % center head
        \rhead{2014/2015} % right head
        \lfoot{} % left foot
        \cfoot{\thepage/\pageref{LastPage}} % center foot
        \rfoot{} % right foot






%%%%%%%%%%%%%%%%%%%%%%%%%
% BEGIN                 %
%%%%%%%%%%%%%%%%%%%%%%%%%
\title{Report of our PRJ project}
\author{Ostiane D'AUGUSTIN and Lucas BOURNEUF}
\begin{document}
        \maketitle % page de titre




%%%%%%%%%%%%%%%%%%%%%%%%%
% SECTION               %
%%%%%%%%%%%%%%%%%%%%%%%%%
\section*{Introduction}
	\paragraph*{}
	The aim of this project is to implement a genetic algorithm which will bring out the connexions between different genes, their regulation networks, and some properties linked to, as well as their resistance to genes knock-out (KO).
	\paragraph*{}
	We will study a famous genes network example, which has been highlighted by Wagner. Wagner's genes network is an artificial genes network calculation model. It has explained all the development and evolution process of genetic regulation networks. It has been first developed by Andreas Wagner in 1996, and then used by several research teams in order to study genes network evolution, genes expression, etc.
	\paragraph*{}
	We will simulate \textit{in silico} a population's evolution over a lot of generations. Each individual is characterized by his genotype (that is to say, his genes networks), and a phenotype, which corresponds to this genes network's expression. The individuals will be submit to an evolutive process by the intermediary of their genes network.




%%%%%%%%%%%%%%%%%%%%%%%%%
% SECTION               %
%%%%%%%%%%%%%%%%%%%%%%%%%
\section{Watchables}
    \paragraph*{}
    Main data generated results of gene KO, for each gene at each generation. 
    A diversity ratio is also computed at each generation.

    \paragraph*{}
     First, we tested our program with the parameters given by the subject, that is to say 200 generations with 300 individuals per generation. As there was no mutation rate given, we varied it from $10^{-1}$ (that is to say, a mutation rate of 1) to $10^{-6}$. We will then study the influence of the mutation rate on a random population of 300 individuals over 200 generations.
    \paragraph*{}
     With those parameters, we can notice the five genes have a similar course : their survivability ratio first fastly increase from 0.7 during the fifty first generations. After this exponentianal phase, the survavibility ratio reach a plateau about 1. 




%%%%%%%%%%%%%%%%%%%%%%%%%
% SECTION               %
%%%%%%%%%%%%%%%%%%%%%%%%%
\section*{Results through graphics}
    \subsection*{Exploitation of the default parameters}
    \paragraph*{}
    \begin{figure}[H] 
      \centering
      \begin{tikzpicture}
          \begin{axis}[
                  %,no mark % retract marks at point location
                  ,xlabel={Generations}
                  ,ylabel={Survivability ratio}
                  ,grid=major % grid on all axis
                  ,axis lines=left
                  ,ymajorgrids
                  ,legend style={legend pos=south east}
                  ,cycle list name=list
              ]
              \addplot table [col sep=comma,x=generationnumber,y=viabilityratio0] {ps200xg300xmr1-10-1.csv};
              \addplot table [col sep=comma,x=generationnumber,y=viabilityratio1] {ps200xg300xmr1-10-1.csv};
              \addplot table [col sep=comma,x=generationnumber,y=viabilityratio2] {ps200xg300xmr1-10-1.csv};
              \addplot table [col sep=comma,x=generationnumber,y=viabilityratio3] {ps200xg300xmr1-10-1.csv};
              \addplot table [col sep=comma,x=generationnumber,y=viabilityratio4] {ps200xg300xmr1-10-1.csv};
              \addplot table [col sep=comma,x=generationnumber,y=diversity]       {ps200xg300xmr1-10-1.csv};
              \legend{Gene 0, Gene 1, Gene 2, Gene 3, Gene 4, Diversity}
          \end{axis}
      \end{tikzpicture}
      \caption{Resistance to gene KO survivability ratio, over 200 generations, with a population of 300 individuals and a mutation rate of 1
      \label{fig:ps200xg300xmr1-10-2.csv}
    \end{figure}
    \paragraph*{}
    We can notice that during the fifty first generations, the resistance to gene KO survivability ratio is still incresing from 0.7 to near to 1, and then, it reach to plateau. This shows that there is no gene wich is absolutly essential for the individual survivability. As a result, the diversity is very close to 1.
    
    
    \begin{figure}[H] 
      \centering
      \begin{tikzpicture}
          \begin{axis}[
                  %,no mark % retract marks at point location
                  ,xlabel={Generations}
                  ,ylabel={Survivability ratio}
                  ,grid=major % grid on all axis
                  ,axis lines=left
                  ,ymajorgrids
                  ,legend style={legend pos=south east}
                  ,cycle list name=list
              ]
              \addplot table [col sep=comma,x=generationnumber,y=viabilityratio0] {ps200xg300xmr1-10-2.csv};
              \addplot table [col sep=comma,x=generationnumber,y=viabilityratio1] {ps200xg300xmr1-10-2.csv};
              \addplot table [col sep=comma,x=generationnumber,y=viabilityratio2] {ps200xg300xmr1-10-2.csv};
              \addplot table [col sep=comma,x=generationnumber,y=viabilityratio3] {ps200xg300xmr1-10-2.csv};
              \addplot table [col sep=comma,x=generationnumber,y=viabilityratio4] {ps200xg300xmr1-10-2.csv};
              \addplot table [col sep=comma,x=generationnumber,y=diversity]       {ps200xg300xmr1-10-2.csv};
              \legend{Gene 0, Gene 1, Gene 2, Gene 3, Gene 4, Diversity}
          \end{axis}
      \end{tikzpicture}
      \caption{Resistance to gene KO survivability ratio, over 200 generations, with a population of 300 individuals and a mutation rate of $10^{-1}$}
      \label{fig:ps200xg300xmr1-10-2.csv}
    \end{figure}
    \paragraph*{}
    In this second graph, the diversity is still very close to 1. However, we can notice that the resistance to gene KO doesn't have the same conduct : gene 0 resistance to gene KO increases very fastly to 1 (about 20 generations), whereas gene 3 has a pretty bad resistance to gene KO, moreover compared to the other genes. For example, since the 120th generation to the  150th one, the resistance to gene 3 KO ration survivability steaily decrease from 0.95 to less than 0.8. This shows that the gene 3 is pretty essential to an individual's survivability.
 

    \begin{figure}[H] 
      \centering
      \begin{tikzpicture}
          \begin{axis}[
                  %,no mark % retract marks at point location
                  ,xlabel={Generations}
                  ,ylabel={Survivability ratio}
                  ,grid=major % grid on all axis
                  ,axis lines=left
                  ,ymajorgrids
                  ,legend style={legend pos=south east}
                  ,cycle list name=list
              ]
              \addplot table [col sep=comma,x=generationnumber,y=viabilityratio0] {ps200xg300xmr1-10-3.csv};
              \addplot table [col sep=comma,x=generationnumber,y=viabilityratio1] {ps200xg300xmr1-10-3.csv};
              \addplot table [col sep=comma,x=generationnumber,y=viabilityratio2] {ps200xg300xmr1-10-3.csv};
              \addplot table [col sep=comma,x=generationnumber,y=viabilityratio3] {ps200xg300xmr1-10-3.csv};
              \addplot table [col sep=comma,x=generationnumber,y=viabilityratio4] {ps200xg300xmr1-10-3.csv};
              \addplot table [col sep=comma,x=generationnumber,y=diversity]       {ps200xg300xmr1-10-3.csv};
              \legend{Gene 0, Gene 1, Gene 2, Gene 3, Gene 4, Diversity}
          \end{axis}
      \end{tikzpicture}
      \caption{Resistance to gene KO survivability ratio, over 200 generations, with a population of 300 individuals and a mutation rate of $10^{-2}$}
      \label{fig:ps200xg300xmr1-10-3.csv}
    \end{figure}
    \paragraph*{}
    In this third graph, the diversity is still very close to 1, but, contrary to the two first graph, it is not a constant, and we can see little variations. However, the resistance to gene KO survivability ratio is very similar to the first one, then we won't develop any more.
    

    \begin{figure}[H] 
      \centering
      \begin{tikzpicture}
          \begin{axis}[
                  %,no mark % retract marks at point location
                  ,xlabel={Generations}
                  ,ylabel={Survivability ratio}
                  ,grid=major % grid on all axis
                  ,axis lines=left
                  ,ymajorgrids
                  ,legend style={legend pos=south east}
                  ,cycle list name=list
              ]
              \addplot table [col sep=comma,x=generationnumber,y=viabilityratio0] {ps200xg300xmr1-10-4.csv};
              \addplot table [col sep=comma,x=generationnumber,y=viabilityratio1] {ps200xg300xmr1-10-4.csv};
              \addplot table [col sep=comma,x=generationnumber,y=viabilityratio2] {ps200xg300xmr1-10-4.csv};
              \addplot table [col sep=comma,x=generationnumber,y=viabilityratio3] {ps200xg300xmr1-10-4.csv};
              \addplot table [col sep=comma,x=generationnumber,y=viabilityratio4] {ps200xg300xmr1-10-4.csv};
              \addplot table [col sep=comma,x=generationnumber,y=diversity]       {ps200xg300xmr1-10-4.csv};
              \legend{Gene 0, Gene 1, Gene 2, Gene 3, Gene 4, Diversity}
          \end{axis}
      \end{tikzpicture}
      \caption{Resistance to gene KO survivability ratio, over 200 generations, with a population of 300 individuals and a mutation rate of $10^{-3}$}
      \label{fig:ps200xg300xmr1-10-4.csv}
    \end{figure}
    \paragraph*{}

    \begin{figure}[H] 
      \centering
      \begin{tikzpicture}
          \begin{axis}[
                  %,no mark % retract marks at point location
                  ,xlabel={Generations}
                  ,ylabel={Survivability ratio}
                  ,grid=major % grid on all axis
                  ,axis lines=left
                  ,ymajorgrids
                  ,legend style={legend pos=south east}
                  ,cycle list name=list
              ]
              \addplot table [col sep=comma,x=generationnumber,y=viabilityratio0] {ps200xg300xmr1-10-5.csv};
              \addplot table [col sep=comma,x=generationnumber,y=viabilityratio1] {ps200xg300xmr1-10-5.csv};
              \addplot table [col sep=comma,x=generationnumber,y=viabilityratio2] {ps200xg300xmr1-10-5.csv};
              \addplot table [col sep=comma,x=generationnumber,y=viabilityratio3] {ps200xg300xmr1-10-5.csv};
              \addplot table [col sep=comma,x=generationnumber,y=viabilityratio4] {ps200xg300xmr1-10-5.csv};
              \addplot table [col sep=comma,x=generationnumber,y=diversity]       {ps200xg300xmr1-10-5.csv};
              \legend{Gene 0, Gene 1, Gene 2, Gene 3, Gene 4, Diversity}
          \end{axis}
      \end{tikzpicture}
      \caption{Resistance to gene KO survivability ratio, over 200 generations, with a population of 300 individuals and a mutation rate of $10^{-4}$}
      \label{fig:ps200xg300xmr1-10-5.csv}
    \end{figure}
    \paragraph*{}
    
        \begin{figure}[H] 
      \centering
      \begin{tikzpicture}
          \begin{axis}[
                  %,no mark % retract marks at point location
                  ,xlabel={Generations}
                  ,ylabel={Survivability ratio}
                  ,grid=major % grid on all axis
                  ,axis lines=left
                  ,ymajorgrids
                  ,legend style={legend pos=south east}
                  ,cycle list name=list
              ]
              \addplot table [col sep=comma,x=generationnumber,y=viabilityratio0] {ps200xg300xmr1-10-6.csv};
              \addplot table [col sep=comma,x=generationnumber,y=viabilityratio1] {ps200xg300xmr1-10-6.csv};
              \addplot table [col sep=comma,x=generationnumber,y=viabilityratio2] {ps200xg300xmr1-10-6.csv};
              \addplot table [col sep=comma,x=generationnumber,y=viabilityratio3] {ps200xg300xmr1-10-6.csv};
              \addplot table [col sep=comma,x=generationnumber,y=viabilityratio4] {ps200xg300xmr1-10-6.csv};
              \addplot table [col sep=comma,x=generationnumber,y=diversity]       {ps200xg300xmr1-10-6.csv};
              \legend{Gene 0, Gene 1, Gene 2, Gene 3, Gene 4, Diversity}
          \end{axis}
      \end{tikzpicture}
      \caption{Resistance to gene KO survivability ratio, over 200 generations, with a population of 300 individuals and a mutation rate of $10^{-5}$}
      \label{fig:ps200xg300xmr1-10-6.csv}
    \end{figure}
    \paragraph*{}

%%%%%%%%%%%%%%%%%%%%%%%%%
% SECTION               %
%%%%%%%%%%%%%%%%%%%%%%%%%
%%%%%%%%%%%%%%%%%%%%%%%%%
% APPENDICES            %
%%%%%%%%%%%%%%%%%%%%%%%%%
%\newpage
%\begin{appendix}
%\section*{Annexes}

%\begin{figure}[h] % place it THERE (add ! for exact placement)
        %\centering
        %\pgfplotstabletypeset[
            %columns={time,wtf,error},
            %col sep=comma,
            %string type,
            %every head row/.style={%
                %before row={\hline
                    %& \multicolumn{2}{c}{useless data} \\
                    %},
                %after row=\hline
            %},
            %every last row/.style={after row=\hline},
            %columns/time/.style={column name=Time, column type=l},
            %columns/error/.style={column name=Primary key, column type=r},
            %columns/wtf/.style={column name=Concentration, column type=r},
        %]{data.csv}
        %\caption{Cool things that will be very useful for someone}
        %\label{ftab:cool}
%\end{figure}      



%\end{appendix}
\end{document}
% END

